%%%%%%%%%%%%%%%%%%%%%%%%%%%%%%%%%%%% PREAMBLES %%%%%%%%%%%%%%%%%%%%%%%%%%%%%%%%%%%%
% sets the base layout
% this class is based on article
\documentclass{_source/ib}

% include the following packages
%\usepackage{amsmath,mathtools}      % useful macros and shortcuts
%\usepackage{booktabs}               % better tables
%\usepackage[f]{esvect}              % better vector symbols
\usepackage{siunitx}                % SI Unit macros
%\usepackage{textcomp,gensymb}       % symbols like degrees, celcius, ...
%\usepackage{pgfplots}               % drawing graphs

% custom macros
\newcommand{\A}{\mathrm A}
\newcommand{\B}{\mathrm B}
\newcommand{\C}{\mathrm C}
\newcommand{\D}{\mathrm D}
\newcommand{\E}{\mathrm E}
\newcommand{\F}{\mathrm F}
\newcommand{\G}{\mathrm G}
\renewcommand{\H}{\mathrm H}
\newcommand{\I}{\mathrm I}
\newcommand{\J}{\mathrm J}
\newcommand{\K}{\mathrm K}
\renewcommand{\L}{\mathrm L}
\newcommand{\M}{\mathrm M}
\newcommand{\N}{\mathrm N}
\renewcommand{\O}{\mathrm O}
\renewcommand{\P}{\mathrm P}
\newcommand{\Q}{\mathrm Q}
\newcommand{\R}{\mathrm R}
\renewcommand{\S}{\mathrm S}
\newcommand{\T}{\mathrm T}
\newcommand{\U}{\mathrm U}
\newcommand{\V}{\mathrm V}
\newcommand{\W}{\mathrm W}
\newcommand{\X}{\mathrm X}
\newcommand{\Y}{\mathrm Y}
\newcommand{\Z}{\mathrm Z}
\newcommand{\dd}{\mathrm d}      % \A gives an upright A and so on

% location of the image files for the layout
% change the directory appropriately if you would like to move the image files
\graphicspath{{source/graphics/}}

% set the name of the exam
\title{Mathematics: applications and interpretation \linebreak Higher level \\ Paper Example}
% set the date of the exam
\date{Wednesday 20 April 2022}
% set the length of the exam
\testlength{50 minutes}

% box options
%   \numberbox draws space for candidate number
%   \namebox draws space for student name
\namebox 

% logos on the front page
% default setting reads source/graphics/schoollogo
% uncomment the line below to use the IB logos instead
%\useiblogos

% draw the grade table on the front page
%   \drawgradetable gives a table with 8 questions per row.
%   \drawgradetable[n] gives a table with n questions per row.
\drawgradetable 

%%%%%%%%%%%%%%%%%%%%%%%%%%%%%%% CONTENT STARTS HERE %%%%%%%%%%%%%%%%%%%%%%%%%%%%%%%
% all content should go below here
\begin{document}
\raggedright% left-align text

% exam instructions on the front page
\begin{instructions}
    \item Write your name in the box above.
    \item Do not open this examination paper until instructed to do so.
    \item A graphic display calculator is required for this paper.
    \item Answer all questions.
    \item Answers must be written within the answer boxes provided.
    \item Unless otherwise stated in the question, all numerical answers should be given exactly or correct to three significant figures.
    \item A clean copy of the \textbf{mathematics: applications and interpretation formula booklet} is required for this paper.
    \item The maximum mark for this examination paper is \textbf{[\totalpoints\ marks]}.% don't change this line
\end{instructions}

% draws the front page
\maketitle

% page 2
\donotwrite% insert 'dont' write here' box
\newpage

% page 3
% put any instructions, guides, etc. here
% use two line-breaks to make a new paragraph
Answers must be written within the answer boxes provided. Full marks are not necessarily awarded for~a~correct answer with no working. Answers must be supported by working and/or explanations. Solutions found from a graphic display calculator should be supported by suitable working. For example, if graphs are used to find a solution, you should sketch these as part of your answer. Where an answer is~incorrect, some marks may be given for a correct method, provided this is shown by written working. You~are therefore advised to show all working.

% questions start here
\begin{questions}
% by default, \question prints the total points on the first line.
% hence, all points for the question itself (not its parts) should be added with \points*, not \points
\question George goes fishing. From experience he knows that the mean number of fish he catches per hour is $1.1$. It is assumed that the number of fish he catches can be modelled by a Poisson distribution. 

On a day in which George spends $8$ hours fishing, find the probability that he will catch more than $9$ fish. \points*{4}

% draw a space for answers
% \makeanswerbox fills the rest of the page
% \makeanswerbox[3cm] is 3cm tall
\makeanswerbox
\newpage% insert a page break, otherwise the next question may break

% $T$ is an italic T. Use an upright T with $\T$ (defined on line 15).
\question The Voronoi diagram below shows three identical cellular phone towers, $\T1$, $\T2$ and $\T3$. A fourth cellular phone tower, $\T4$ is located in the shaded region. The dashed lines in the diagram below represent the edges in the Voronoi diagram.

% \SI and \num are convenient macros for scientific notation (from package siunitx)
% for example, \SI{3e8}{\m\per\s} gives $3\times 10^{8} m s^{-1}$ and 
% \num{1.2e-14} gives $1.2\times 10^{-14}$.
% for more information visit https://ctan.org/pkg/siunitx
Horizontal scale: $1$ unit represents \SI{1}{\km}. \linebreak
Vertical scale: $1$ unit represents \SI{1}{\km}.
% [width=0.66\textwidth] makes the image fill 2/3 of the horizontal space
\insertimage[width=0.66\textwidth]{fig/example-pic-ib-1}

Tim stands inside the shaded region. \points{2}
\begin{parts}
    \part Explain why Tim will receive the strongest signal from tower $\T4$. \points{1}
\end{parts}
% Any mid-question explanations should sit outside \begin{parts} ... \end{parts}
% for the proper indentation
Tower $\T2$ has coordinates $(−9 , 5)$ and the edge connecting vertices $\A$ and $\B$ has equation $y = 3$.
\begin{parts}
    \part Write down the coordinates of tower $\T4$. \points{2}
\end{parts}
% same as above
Tower $\T1$ has coordinates $(-13,3)$.
\begin{parts}
    \part Find the gradient of the edge of the Voronoi diagram between towers $\T1$ and $\T2$. \points{3}
\end{parts}

% insert some vertical space for better organisation
\bigskip
% \continue inserts a page break with a reminder:
%    (question continues in the next page) and (question continued)
\continue
% the next page only has an answer box
\makeanswerbox
\newpage

% percent symbol is reserved for comments
% use \% or \SI{2}{\percent} instead
\question Charlie and Daniella each began a fitness programme. On day one, they both ran \SI{500}{\m}. On each subsequent day, Charlie ran \SI{100}{\m} more than the previous day whereas Daniella increased her distance by \SI{2}{\percent} of the distance ran on the previous day.

\begin{parts}
    \part Calculate how far
    % put subparts
    \begin{subparts}
        \subpart Charlie ran on day $20$ of his fitness programme.
        \subpart Daniella ran on day $20$ of her fitness programme. \points{5}
    \end{subparts}
\end{parts}

On day $n$ of the fitness programmes Daniella runs more than Charlie for the first time.

\begin{parts}
    \part Find the value of $n$. \points{3}
\end{parts}

\makeanswerbox

% questions end here
\end{questions}

% adds an appropriate number of \donotwrite pages
% so the total number of pages is a multiple of four
\fillbooklet

% all content should be placed above here
\end{document}
