%%%%%%%%%%%%%%%%%%%%%%%%%%%%%%%%%%%% PREAMBLES %%%%%%%%%%%%%%%%%%%%%%%%%%%%%%%%%%%%
% sets the base layout
% this class is based on article
\documentclass{src/hsc}

% include the following packages
%\usepackage{amsmath,mathtools}      % useful macros and shortcuts
%\usepackage{booktabs}               % better tables
%\usepackage[f]{esvect}              % better vector symbols
\usepackage{siunitx}                % SI Unit macros
%\usepackage{textcomp,gensymb}       % symbols like degrees, celcius, ...
%\usepackage{pgfplots}               % drawing graphs

% custom macros
% macros-mathrm.tex
%
% creates (or replaces) the macros for upright alphabets.
% you can write triangle ABC as $\triangle \A\B\C$ and
% dy/dx as $\frac{\dd y}{\dd x}$, and so on.
% note: it replaces five macros already defined in (La)TeX
\newcommand{\A}{\mathrm A}
\newcommand{\B}{\mathrm B}
\newcommand{\C}{\mathrm C}
\newcommand{\D}{\mathrm D}
\newcommand{\E}{\mathrm E}
\newcommand{\F}{\mathrm F}
\newcommand{\G}{\mathrm G}
\renewcommand{\H}{\mathrm H}
\newcommand{\I}{\mathrm I}
\newcommand{\ii}{\mathrm i}
\newcommand{\J}{\mathrm J}
\newcommand{\K}{\mathrm K}
\renewcommand{\L}{\mathrm L}
\newcommand{\M}{\mathrm M}
\newcommand{\N}{\mathrm N}
\renewcommand{\O}{\mathrm O}
\renewcommand{\P}{\mathrm P}
\newcommand{\Q}{\mathrm Q}
\newcommand{\R}{\mathrm R}
\renewcommand{\S}{\mathrm S}
\newcommand{\T}{\mathrm T}
\newcommand{\U}{\mathrm U}
\newcommand{\V}{\mathrm V}
\newcommand{\W}{\mathrm W}
\newcommand{\X}{\mathrm X}
\newcommand{\Y}{\mathrm Y}
\newcommand{\Z}{\mathrm Z}
\newcommand{\dd}{\mathrm d}      % \A gives an upright A and so on

% location of the image files for the layout
% change the directory appropriately if you would like to move the image files
\graphicspath{{src/graphics/}}

\schoolname{School Name}
\examyear{2022}
\examtitle{FINAL EXAMINATION}
\examsubject{Physics}

\begin{document}

%%% SECTION I BEGIN

% title
\sectiononeinfo{
    \begin{infobox}{General Instructions}
        \begin{infoitems}
            \item Reading time -- 5 minutes
            \item Working time -- 3 hours
            \item Write using black pen
            \item Draw diagrams using pencil
            \item Calculators approved by NESA may be used
            \item A data sheet, formulae sheet and Periodic Table are provided at the back of this paper
        \end{infoitems}
    \end{infobox}

    \begin{infobox}{Total marks:\linebreak \totalpoints}
        \textbf{Section I -- 20 marks} (pages 2--24)
        \begin{infoitems}
            \item Attempt Questions 1--20
            \item Allow about 35 minutes for this section
        \end{infoitems}\medskip

        \textbf{Section II -- 80 marks} (pages 17--36)
        \begin{infoitems}
            \item Attempt Questions 21--35
            \item Allow about 2 hours and 25 minutes for this section
        \end{infoitems}
    \end{infobox}
}
\sectiononetitle

\examsection{Section I}


{\raggedright
    \textbf{%
        20 marks \linebreak
        Attempt Questions 1--20 \linebreak
        Allow about 35 minutes for this section
    }

    Use the multiple-choice answer sheet for Questions 1--20.

    \vspace{-0.6cm}\hrulefill
}

\begin{mcquestions}
    \question\label{mcqfirst} A marble is rolled off a horizontal bench and falls to the floor.
    
    \includecenteredgraphics[height=4.5cm]{fig/example-hsc-1}

    Rolling the marble at a slower speed would

    \begin{choices}
        \choice increase the range.
        \choice decrease the range.
        \choice increase the time of flight.
        \choice decrease the time of flight.
    \end{choices}

    \question A positively charged particle is moving at velocity, $v$, in an electric field as shown.

    \includecenteredgraphics[height=3.4cm]{fig/example-hsc-2}

    What is the direction of the force acting on the particle due to the electric field?

    \begin{choices}
        \choice Into the page
        \choice Out of the page
        \choice Up the page
        \choice Down the page
    \end{choices}

    \clearpage

    \question Which of the following is NOT a fundamental particle in the Standard Model of matter?

    \begin{choices}
        \choice Electron
        \choice Gluon
        \choice Muon
        \choice Proton
    \end{choices}

    \question\label{mcqlast} An astronaut is travelling towards Earth in a spaceship at $0.8c$. At regular intervals, a radio pulse is sent from the spaceship to an observer on Earth.

    Which quantity would the astronaut and the observer measure to be the same?

    \begin{choices}
        \choice Length of the spaceship
        \choice Speed of the radio pulses
        \choice Momentum of the astronaut
        \choice Time interval between the radio pulses
    \end{choices}
\end{mcquestions}

\fillbooklet{\blankpage}

%%% SECTION I END

%%% SECTION II BEGIN

% title
\sectiontwoinfo{%
    \textbf{%
        80 marks \linebreak
        Attempt Questions 21--35 \linebreak
        Allow about 2 hours and 25 minutes for this section
    }
    \vspace{1.2cm}

    \begin{infobox}{Instructions}
        \begin{infoitems}[itemsep=2ex]
            \item Write your Centre Number and Student Number at the top of this page.
            \item Answer the questions in the spaces provided. These spaces provide guidance for the expected length of response.
            \item Show all relevant working in questions involving calculations.
            \item Extra writing space is provided at the back of this booklet. \linebreak
            If you use this space, clearly indicate which question you are answering.
        \end{infoitems}
    \end{infobox}

}
\sectiontwotitle

\begin{questions}
    \question\label{safirst} A DC motor is constructed from a single loop of wire with dimensions $\SI{0.10}{\m}$. The magnetic field strength is $\SI{0.40}{\T}$ and a current of $\SI{14}{\A}$ flows through the loop.

    \includecenteredgraphics[height=4.95cm]{fig/example-hsc-3}

    \begin{parts}
        \part\points{2} Calculate the magnitude of the maximum torque produced by the motor.

        \answerbox[4]

        \part\points{2} Describe how the magnitude of the torque changes as the loop moves through half a rotation from the position shown. 

        \answerbox[4]
    \end{parts}

    \questionbreak

    How do the results from

    \question\label{salast} Another question

\end{questions}

\clearpage
%%% SECTION II END

\writingspace
\writingspace

\end{document}