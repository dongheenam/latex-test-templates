%%%%%%%%%%%%%%%%%%%%%%%%%%%%%%%%%%%% PREAMBLES %%%%%%%%%%%%%%%%%%%%%%%%%%%%%%%%%%%%
% sets the base layout
% this class is an extension of article
\documentclass{_source/quiz}

% packages already loaded in the class:
% (default)         geometry, [inline]enumitem, fancyhdr, iftex, paracol
% (XeTeX or LuaTeX) fontspec, unicode-math 
% (other compilers) inputenc, newtxtest, newtxmath
%
% include the following packages
\usepackage{amsmath,mathtools}       % for useful macros and shortcuts
%\usepackage{siunitx}                 % for SI Unit macros
%\usepackage{textcomp,gensymb}        % symbols like degrees, celcius, ...
%\usepackage{tikz,pgfplots}           % for drawing graphs and diagrams

% set the name of the quiz
\title{Example Quiz}

% default behaviour is that exam sections do not reset the question numbers
% \sectionresetsquestion makes every section starts with Question 1
%\sectionresetsquestion

%%%%%%%%%%%%%%%%%%%%%%%%%%%%%%% CONTENT STARTS HERE %%%%%%%%%%%%%%%%%%%%%%%%%%%%%%%
% all content should go below here
\begin{document}

% display the name of the quiz and draw spaces for student name and total marks
% note: the marks shown here is the sum of all marks given in the questions and parts below
\maketitle

% put any instructions, guides, etc. here
% use two line-breaks to make a new paragraph
You can use a scientific calculator to solve the questions.

Time given: 15 minutes.
Good luck!

% \examsection breaks questions into different sections.
%\examsection{Basic Algebra}

% questions start here
\begin{questions}

% basic question
\question What is the coefficient of $x$ in $x^2 - x + 2$?

% question with points
\question Expand $3(x^2-2x)$. \points{5}

% question with hidden points
\question Collect the like terms for $x^2 + 3x - 2x - 7$. \points*{2}

% put a vertical space between questions
% you can put lengths in px, cm, mm, in, em.
\question Expand and simplify $(x+2)(x+3)-x^2$. \points{4}
\vspace{2cm}

% if you want equal spacing between questions, use \vspace{\stretch{1}}
% the number inside \stretch determines the relative spacing between the questions
\question Simplify this algebraic fraction: $\dfrac{2(x+3)(x-2)}{x-2}$. \points{4}
\vspace{\stretch{1}}

\question Fully factorise $x^2 - 5x + 6$. \points{4}
\vspace{\stretch{1}}

% a page break
\newpage

% multiple-choice question
\question Which expression is equal to $x^2 - x$?  \points*{1}
\begin{choices}
    \choice $x(x+1)$
    \choice $x(x-1)$
    \choice $x^2 - 2x - x$
    \choice $x^2 + 2x - x$
\end{choices}

% MC question with inline choices
\question If you expand $(2x-1)(x+3)$, what will be the coefficient of $x$?

\begin{choices*}
    \choice -6
    \choice -3
    \choice 5
    \choice 6
\end{choices*}

% question with parts
\question Simplify: \points{4}
\begin{parts}
    \part $\dfrac{3x}{x}$
    \vspace{1cm}
    \part $\dfrac{3x^2y}{12y^3}$
    \vspace{1cm}
\end{parts}

% parts with points
% use \question* instead of \question to show the total points for the question.
\question* Use index laws to simplify the following.
\begin{parts}
    \part $x^2 y^3 \times x y^2 z^3 \times x^5 z^2$  \points{3}
    \part $\dfrac{x^{10} y^{12}}{x^3 y^{-2}}$ \points{4}
\end{parts}
\vspace{2cm}

% parts across multiple columns
\question* Simplify:
\begin{mcparts}[2]
    \part $3a + 8a$ \points{1}
    \part $5x + 2x + 4x$ \points{1}
    \vspace{2cm}
    \part $4y-3y+8$ \points{1}
    \part $7x+5-4x$ \points{1}
    \vspace{2cm}
    \part $2-5m-m$ \points{1}
    \part $y^2+2y+3y-1$ \points{2}
\end{mcparts}

% insert a page break
\newpage

% insert an image
% usage: \insertimage[width=...]{address-to-file}
% note: this is a custom macro.
%       LaTeX default is \includegraphics[options]{address-to-file}
\question What is the most specific name of this shape? \points{1}

\insertimage[width=0.4\textwidth]{_fig/example-pic-quiz-1}
\vspace{1cm}

% insert an image on the right side
% usage: \insertimageonright[width=...]{address-to-file}
% note: the text will not wrap around the image, so be careful about it.
\question* For the diagram on the right, \insertimageonright[width=4cm]{_fig/example-pic-quiz-1}
\begin{parts}
    \part find $x$. \points{1}
    \vspace{2cm}
    \part find the angle opposite the side that is 15 cm long. \points{2}
    \vspace{2cm}
\end{parts}

% specify answers to the questions
\question Rationalise the denominator of $\dfrac{2}{\sqrt{1+x}}$, where $x \ge -1$. \points{2}
\begin{answers}
$\dfrac{2 \sqrt{x+1}}{x+1}$
\end{answers}
\vspace{1cm}

% the answer environment works per question, not per part
\question* Fully factorise: 
\begin{parts}
    \part $x^2 + 3x + 2$ \points{3}
    \part $2x^2 - 14x + 20$ \points{2}
    \part $2x^2 - x - 1$ \points{4}
\end{parts}

\begin{answers}
\begin{parts}
    \part $(x+1)(x+2)$
    \part $2(x-2)(x-5)$
    \part $(2x+1)(x-1)$
\end{parts}
\end{answers}

% questions end here
\end{questions}

% remove this command if you don't want to print the answers
\showallanswers

% all content should be placed above here
\end{document}
