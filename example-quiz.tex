%%%%%%%%%%%%%%%%%%%%%%%%%%%%%%%%%%%% PREAMBLES %%%%%%%%%%%%%%%%%%%%%%%%%%%%%%%%%%%%
% sets the base layout
% this class is an extension of examdoc
\documentclass{source/quiz}

% packages already loaded in the class:
% (default)         geometry, iftex, paracol
% (XeTeX or LuaTeX) fontspec, unicode-math 
% (other compilers) inputenc, newtxtest, newtxmath
%
% include the following packages
\usepackage{amsmath,mathtools}       % for useful macros and shortcuts
%\usepackage{siunitx}                 % for SI Unit macros
%\usepackage{textcomp,gensymb}        % symbols like degrees, celcius, ...
%\usepackage{tikz,pgfplots}           % for drawing graphs and diagrams

% set the name of the quiz
\title{Example Quiz}

%%%%%%%%%%%%%%%%%%%%%%%%%%%%%%% CONTENT STARTS HERE %%%%%%%%%%%%%%%%%%%%%%%%%%%%%%%
% all content should go below here
\begin{document}

% display the name of the quiz and draw spaces for student name and total marks
% note: the marks shown here is the sum of all marks given in the questions and parts below
\maketitle

% put any instructions, guides, etc. here
% use two line-breaks to make a new paragraph
You can use a scientific calculator to solve the questions.

Time given: 15 minutes.
Good luck!

% questions start here
\begin{questions}

% make a new question
% usage: \question instructions
\question What is the coefficient of $x$ in $x^2 - x + 2$?

% you can also assign a mark for the question.
% put \droppoints at the end of the sentence to display the marks. Otherwise they are only internally used.
% usage: \question[marks-for-question] instructions \droppoints
\question[2] Collect the like terms for $x^2 + 3x - 2x - 7$.
\question[2] Expand $3(x^2-2x)$. \droppoints

% put a vertical space between questions using \vspace{length}
% you can put lengths in px, cm, mm, in, em.
% if you want equal spacing between questions, use \vspace{\stretch{1}}
% the number inside \stretch determines the relative spacing between the questions
\question[4] Expand and simplify $(x+2)(x+3)-x^2$. \droppoints
\vspace{2cm}

\question[4] Simplify this algebraic fraction: $\dfrac{2(x+3)(x-2)}{x-2}$. \droppoints
\vspace{\stretch{1}}

\question[4] Fully factorise $x^2 - 5x + 6$. \droppoints
\vspace{\stretch{1}}

% insert a page break with \newpage (same as Ctrl + Enter in Word)
\newpage

% make a multiple-choice question with the following syntax.
% \begin{choices}
%     \choice option
%     ...
% \end{choices}
\question[1] Which expression is equal to $x^2 - x$?
\begin{choices}
    \choice $x(x+1)$
    \choice $x(x-1)$
    \choice $x^2 - 2x - x$
    \choice $x^2 + 2x - x$
\end{choices}
\vspace{1cm}

% put the choices into a single line
% usage: \begin{oneparchoices} ... \end{oneparchoices}
\question[1] If you expand $(2x-1)(x+3)$, what will be the coefficient of $x$?
\vspace{0.5cm}

\begin{oneparchoices}
    \choice -6
    \choice -3
    \choice 5
    \choice 6
\end{oneparchoices}
\vspace{1cm}

% make parts of a question
% insert vertical space using \vspace
% \begin{parts}
%     \part instruction
%     ...
% \end{parts}
\question[4] Simplify: \droppoints
\begin{parts}
    \part $\dfrac{3x}{x}$
    \vspace{1cm}
    \part $\dfrac{3x^2y}{12y^3}$
    \vspace{1cm}
\end{parts}

% you can also assign marks for each part.
% note: don't put any marks for the question as they will be counted twice
%       (\question[3] \part[2] \part[1] is worth 6 marks)
% note: use \droptotalpoints instead of \droppoints to add up the marks of the parts
\question Use index laws to simplify the following. \droptotalpoints
\vspace{1cm}
\begin{parts}
    \part[3] $x^2 y^3 \times x y^2 z^3 \times x^5 z^2$ \droppoints
    \vspace{1cm}
    \part[4] $\dfrac{x^{10} y^{12}}{x^3 y^{-2}}$ \droppoints 
    \vspace{1cm}
\end{parts}

% put parts across multiple columns
% usage: \begin{mcparts}[number-of-columns]
%            \part \instruction
%            ...
%        \end{mcparts}
% note: this is a custom macro.
% note: compatible with \droppoints, but the margins become weird
\question[6] Simplify: \droptotalpoints
\begin{mcparts}[2]
    \part $3a + 8a$
    \part $5x + 2x + 4x$
    \vspace{2cm}
    \part $4y-3y+8$
    \part $7x+5-4x$
    \vspace{2cm}
    \part $2-5m-m$
    \part $y^2+2y+3y-1$
\end{mcparts}

\newpage

% insert an image
% usage: \insertimage[width=...]{address-to-file}
% note: this is a custom macro.
%       the LaTeX default is \includegraphics[options]{address-to-file}
\question[1] What is the most specific name of this shape? \droppoints

\insertimage[]{fig/example-pic1}
\vspace{1cm}

% insert an image on the right side
% usage: \insertimageonright[width=...]{address-to-file}
% note: the text will not wrap around the image, so be careful about it.
\question For the diagram below, \droptotalpoints

\insertimageonright[]{fig/example-pic1}
\begin{parts}
    \part[1] find $x$.
    \vspace{2cm}
    \part[2] find the angle opposite the side that is 15 cm long.
    \vspace{2cm}
\end{parts}

% specify the answers to the questions
% usage: \begin{answer} ... \end{answer}
% put \showallanswers at the end of the document
%   to print the answers to the questions on the last page.
% note: this is a custom macro.
\question[2] Rationalise the denominator of $\dfrac{2}{\sqrt{1+x}}$, where $x \ge -1$. \droppoints
\begin{answer}
$\dfrac{2 \sqrt{x+1}}{x+1}$
\end{answer}
\vspace{1cm}

\question Which one of those cannot be simplified as a multiple of $\sqrt{2}$?
\begin{choices}
    \choice $\sqrt{8}$
    \choice $\sqrt{12}$
    \choice $\sqrt{18}$
    \choice $\sqrt{50}$
\end{choices}
\begin{answer}
B
\end{answer}
\vspace{1cm}

% the answer environment works per question, not per part
\question Fully factorise: \droptotalpoints
\begin{mcparts}[2]
    \part[3] $x^2 + 3x + 2$ \vspace{1cm}
    \part[2] $2x^2 - 14x + 20$
    \part[4] $2x^2 - x - 1$
\end{mcparts}
\begin{answer}
(a) $(x+1)(x+2)$
(b) $2(x-2)(x-5)$
(c) $(2x+1)(x-1)$
\end{answer}

% questions end here
\end{questions}
% place the note at the bottom right of the current page
% you can change the content inside { ... }
\vfill\flushright{ \textbf{END OF QUIZ} }
% reference to the last page of the quiz to count the number of pages
\label{lastpage}
% remove this command if you don't want to print the answers
\showallanswers

% all content should be placed above here
\end{document}
