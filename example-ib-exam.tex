%%%%%%%%%%%%%%%%%%%%%%%%%%%%%%%%%%%% PREAMBLES %%%%%%%%%%%%%%%%%%%%%%%%%%%%%%%%%%%%
% sets the base layout
% this class is an extension of examdoc
\documentclass{source/ib-exam}

% packages already loaded in the class:
% (default)         enumitem, geometry, iftex, tikz(with library calc)
% (XeTeX or LuaTeX) fontspec, unicode-math 
% (other compilers) inputenc, newtxtest, newtxmath
%
% include the following packages
\usepackage{amsmath,mathtools}      % useful macros and shortcuts
\usepackage{booktabs}               % better tables
\usepackage[f]{esvect}              % better vector symbols
\usepackage{siunitx}                % SI Unit macros
%\usepackage{textcomp,gensymb}       % symbols like degrees, celcius, ...
%\usepackage{pgfplots}               % drawing graphs

% custom macros
\newcommand{\A}{\mathrm A}
\newcommand{\B}{\mathrm B}
\newcommand{\C}{\mathrm C}
\newcommand{\D}{\mathrm D}
\newcommand{\E}{\mathrm E}
\newcommand{\F}{\mathrm F}
\newcommand{\G}{\mathrm G}
\renewcommand{\H}{\mathrm H}
\newcommand{\I}{\mathrm I}
\newcommand{\J}{\mathrm J}
\newcommand{\K}{\mathrm K}
\renewcommand{\L}{\mathrm L}
\newcommand{\M}{\mathrm M}
\newcommand{\N}{\mathrm N}
\renewcommand{\O}{\mathrm O}
\renewcommand{\P}{\mathrm P}
\newcommand{\Q}{\mathrm Q}
\newcommand{\R}{\mathrm R}
\renewcommand{\S}{\mathrm S}
\newcommand{\T}{\mathrm T}
\newcommand{\U}{\mathrm U}
\newcommand{\V}{\mathrm V}
\newcommand{\W}{\mathrm W}
\newcommand{\X}{\mathrm X}
\newcommand{\Y}{\mathrm Y}
\newcommand{\Z}{\mathrm Z}
\newcommand{\dd}{\mathrm d}      % \A gives an upright A and so on

% set the name of the exam
\title{Mathematics: applications and interpretation \linebreak Higher level \\[1ex] Year 11 Term 1 Test}
% set the date of the exam
\date{Wednesday 20 April 2022}
% set the length of the exam
\testlength{50 minutes}
% box options
%\numberbox     % uncomment this line to show space for the candidate number
\namebox       % uncomment this line to show space for the student name
% use the IB logos (located source/fig/iblogo_1 and _2) instead of placeholders
\useiblogos    % uncomment this line to use the real IB logos 

%%%%%%%%%%%%%%%%%%%%%%%%%%%%%%% CONTENT STARTS HERE %%%%%%%%%%%%%%%%%%%%%%%%%%%%%%%
% all content should go below here
\begin{document}
\raggedright

\begin{instructionslist}
    \item Write your name in the box above.
    \item Do not open this examination paper until instructed to do so.
    \item A graphic display calculator is required for this paper.
    \item Answer all questions.
    \item Answers must be written within the answer boxes provided.
    \item Unless otherwise stated in the question, all numerical answers should be given exactly or correct to three significant figures.
    \item A clean copy of the \textbf{mathematics: applications and interpreta
    tion formula booklet} is required for this paper.
    \item The maximum mark for this examination paper is \textbf{[\numpoints\ marks]}.% don't change this line
\end{instructionslist}

% display the name of the quiz and draw spaces for student name and total marks
% note: the marks shown here is the sum of all marks given in the questions and parts below
\maketitle

\donotwrite
\newpage

% put any instructions, guides, etc. here
% use two line-breaks to make a new paragraph
Answers must be written within the answer boxes provided. Full marks are not necessarily awarded for a correct answer with no working. Answers must be supported by working and/or explanations. Solutions found from a graphic display calculator should be supported by suitable working. For example, if graphs are used to find a solution, you should sketch these as part of your answer. Where an answer is incorrect, some marks may be given for a correct method, provided this is shown by written working. \linebreak You are therefore advised to show all working.

% questions start here
\begin{questions}

\question Let $f(x)=2x+3$ and $g(x)=\sqrt{x}-3$.

\begin{parts}
    \part[1] Find $(g\circ f)(x)$. \droppoints
    \part[3] State the natural domain of $(g\circ f)(x)$. \droppoints
    \part[2] Determine the $y$-intercept of the graph of $y=(g\circ f)(x)$. \droppoints
\end{parts}

\makeanswerbox
\newpage

\question The following table gives the mass (in kilograms) of the three particles which make up an atom.

\begin{center}
\begin{tabular}{lr}
    \toprule
    \textbf{Particle} & \textbf{Mass (kg)} \\
    \midrule
    Neutron & \num{1.675e-27} \\
    Proton & \num{1.673e-27} \\ 
    Electron & \num{9.109e-31} \\ 
    \bottomrule
\end{tabular}
\end{center}

\begin{parts}
    \part[1] A helium atom consists of two protons, two neutrons and two electrons. Find the mass of a helium atom, giving your answer correct to three significant figures and in the form $a \times 10^k$, where $1\le a<10$, $k\in\mathbb{Z}$. \droppoints
    \part[1] Determine the ratio of electron mass to neutron mass, giving your answer in the form $1:x$ where $x$ is accurate to three significant figures. \droppoints
    \part[2] Calculate the percentage error, correct to three significant figures in taking the mass of an electron to be approximately $\SI{1e-20}{\kg}$. \droppoints
\end{parts}

\makeanswerbox
\newpage

\question 
\begin{parts}
    \part[2] Express $\dfrac{2}{2\sqrt{7}+5}$ in simplest form with a rational denominator.\droppoints
    \part[2] Simplify $\dfrac{x^7 \times x^{-3}}{x^{-6}}$. \droppoints
    \part[3] Solve for $x$ in $x^{-2}=3^{-1}+4^{-1}$. \droppoints
\end{parts}

\makeanswerbox
\newpage

\question The graph of the quadratic function $f(x) = \dfrac{1}{2} (x-2)(x+8)$ intersects the $y$-axis at $(0, c)$.
\begin{parts}
    \part[2] Find the value of $c$. \droppoints
    \part[1] given the vertex of the graph is $(-3, -12.5)$, write down the equation for the axis of symmetry. \droppoints
    \part[4] Solve the equation $f(x) = 12$. \droppoints
\end{parts}

\makeanswerbox
\newpage

\question[4] Consider a function $f(x)$ for $-2\le x\le 2$. The following diagram shows the graph of $y=f(x)$.

\insertframedimage[width=0.5\textwidth]{fig/example-pic2}

On the grid above, draw the graph of $f^{-1}$.

\newpage

\question A king rules a small mountain kingdom, which is in the form of a square of length $4$ kilometres. The square is described by the co-ordinate system $0≤x≤4$, $0≤y≤4$.

The king has four adult children, each of which has a luxury palace located at the points $(1,1)$, $(3,1)$, $(1,3)$, $(3,3)$. Each child owns all the land that is nearer their palace than any other palace.

\begin{parts}
    \part[2] Draw a Voronoi diagram in the grid below to represent this information. \droppoints
    
    \insertframedimage[width=0.4\textwidth]{fig/example-pic3}
    
    \expl{ The king has a fifth (youngest) child who is now just growing up. He installs her in a new palace situated at point $(2,2)$. The rule about ownership of land is then reapplied. }
    
    \part[2] Draw a new Voronoi diagram in the grid below to represent this new situation. \droppoints
    
    \insertframedimage[width=0.4\textwidth]{fig/example-pic3}
    
    \continue
    
    \part[1] State what the shape of the land, owned by the youngest child, is. \droppoints
    \part[3] Find the area of the youngest child’s land. \droppoints
    \part[1] Find how much land an older child has lost. \droppoints
    \part[2] State, with a reason, if all five children now own an equal amount of land. \droppoints
\end{parts}

\makeanswerbox
\newpage

\question Below shows the graph of $L_1:x-5y+5=0$ and Point $\A$, which is the $y$-intercept of $L_1$.

\insertimage{fig/example-pic4}

\begin{parts}
    \part[2] Show that $\B(10,3)$ is on $L_1$. \droppoints
    \part[6] Find the equation of the perpendicular bisector of $[\A\B]$. \droppoints
\end{parts}

\makeanswerbox

% questions end here
\end{questions}

\newpage
\donotwrite
\newpage
\donotwrite 

% all content should be placed above here
\end{document}
